% !TEX root = ../main.tex

%\chapter{Fazit}
\chapter{Conclusion}
\label{sect:conclusion}
		
% Gehe auf Ziel, Aufgabenstellung aus Einleitung ein und wiederhole zusammenfassend, wie die Aufgabenstellung erfüllt wurde und was die Ergebnisse sind.
% Adress the aim and the conceptual formulation of the assignment from the introduction directly and summarise how the formulated goals were reached (or not).

The aim of this document was to provide students at the end of their studies with a template for their written thesis.
Due to the common lack of experience in the field of academic writing this work is intended to provide a template for structured writing.
Therefore, this entire document is structured as a thesis should usually be.

Moreover it provides some hints how citations should be used and how figures should be dealt with to achieve high quality versions of latex documents.
In contrary to this sentence, a conclusion should not introduce new information, about the topics discussed before, which has not yet been presented.

The following (optional) section provides some further ideas for potential extension of this work.

%\section{Ausblick} %(optional)
\section{Future Work} % (optional)

% Wie könnte es weiter gehen? Dieser Abschnitt kann auch in einem eigenen Kapitel vorhergehen.
% Short description how this work could be pursued. This can be done in a separate chapter preceding the conclusion, too.

There are many possible ways how this short document could be extended in the future.
One may think of additional explanations regarding latex and its use, with the extend to an entire latex tutorial.
A further extension could be the definition of helpful latex commands or a documentation on commonly used latex commands and packages.